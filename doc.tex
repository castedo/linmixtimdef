% This file contains the document content that should be used to generate JATS XML, HTML and PDF formats

\addbibresource{references.bib}

\begin{document}


\newcommand{\dom}[1]{\operatorname{dom}{ #1}}
\newcommand{\mathstop}{\text{ .}}

\newcommand{\Gam}{\mathrm{Gam}}
\newcommand{\Loc}{\mathrm{Loc}}
\newcommand{\Lin}{\mathrm{Lin}}
\newcommand{\Mate}{\mathrm{Mate}}
\newcommand{\Par}{\mathrm{Par}}
\newcommand{\Fert}{\mathrm{Fert}}
\newcommand{\Mrlt}{\mathrm{Mrlt}}


\begin{abstract}
\textbf{STAGE:} Early Draft

\textbf{DOCUMENT TYPE:} Mathematical Definition

This document provides a formal mathematical definition of lineal
admixture time. For an introduction and non-mathematical definition see
\href{https://perm.pub/DZFCt68peNNajZ34WtZni9VYxzo/0}{Lineal admixture time: an interdisciplinary definition}:
\\ (\href{https://perm.pub/DZFCt68peNNajZ34WtZni9VYxzo/0}{perm.pub/DZFCt68peNNajZ34WtZni9VYxzo/0}).
\end{abstract}

\section{Introduction}

This document mathematically defines the general meaning of \emph{lineal admixture time}.
A simple special case of \emph{lineal admixture time} is defined non-mathematically
for a broad inter-disciplinary audience in
\href{https://perm.pub/DZFCt68peNNajZ34WtZni9VYxzo/0}{Lineal admixture time: an interdisciplinary definition}
\cite{dsi:DZ/0}.
Some of the benefits of the more general mathematical meaning are:
\begin{enumerate}
\item the option to condition on specific genetic regions (e.g. X chromosome), and
\item compatibility with certain convenient mathematical models such as certain stationary
processes.
\end{enumerate}


\section{Definition}

We formally define lineal admixture time in terms of a
\href{https://popgen.es/0iV47kWzQAuyONrIDG538k3x3Qc/0.3/}{gametic
lineage space}
\cite{dsi:0i/0.3}
\[
(\Loc, (\Gam, \Mate, \Par, \Fert), \Lin)
\mathstop
\]

This formal definition of lineal admixture time also requires:

\begin{enumerate}
\item
  a reference observation time \(\tau\) (usually the present), and
\item
  an isolate categorization \(c\) (e.g.~to ancestral groups).
\end{enumerate}

An isolate categorization \(c\) assigns a subset of non-admixed zygotes
to one or more genetic isolates. A formal definition is given later.

Given isolate categorization \(c\) of zygotes in \(\Mate\), and lineage
\(d\), the \emph{Most Recent Lineal Transition} is
$$
\Mrlt(d,c) := \sup\{ \Fert(\Mate_*(g)) : g \in d, c(\Mate_*(g)) \not= c(\Par(g)) \}
$$

Given an isolate categorization \(a\) and observation time \(\tau\),
lineal admixture time is the random variable
$$
\tau - \Mrlt(L, a)
$$
where $L$ is a random lineage extant at time $\tau$.


\subsection{Isolate Categorization}

We assume a given gametic lineage space
$$
(\Loc, (\Gam, \Mate, \Par, \Fert), \Lin)
\mathstop
$$

An isolate categorization is an assignment $a$ of a subset of zygotes
from $\Mate$ to ancestral categories such that for all zygotes
$z \in \Mate$,

\begin{itemize}
\item
  for all \(z \in \dom{a}, i \in \{0, 1\}, z_i \in \dom{Par}\),
  \(a(\Par(z_i)) = a(z)\), and
\item
  if \(a(\Par(z_0)) = a(\Par(z_1))\) then \(z \in \dom{a}\) (assuming
  \(z_0\) and \(z_1\) are child gametes and their parents are assigned
  to an isolate).
\end{itemize}


\printbibliography % used by LaTeX but not pandoc
\end{document}
